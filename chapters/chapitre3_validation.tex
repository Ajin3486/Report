% Chapter 3: Validation Strategy (~3 pages)

\section{Domain Randomization Approach}
% To complete (~1 page):
% - Context: Time constraint prevents real robot validation
% - Strategy: Test policy under diverse randomized conditions
% - Justification: If robust to varied sim conditions, likely to transfer to real
% - Domain randomization parameters (more extreme than training):
%   * Sensor noise: σ = [0.5mm, 5mm] (vs training: [0.5mm, 2mm])
%   * Calibration error: offset = [0, 5cm] (vs training: [0, 3cm])
%   * Dynamics: friction/mass ±20% (vs training: ±15%)
% - Literature support: Cite 1-2 papers on domain randomization for sim-to-real

\section{Test Protocols}
% To complete (~2 pages):
% - Test set: 100 random trajectories (unseen during training)
% - Metrics: Success rate (error < 5mm), RMS error, max error

\subsection{Test A: Baseline Performance}
% - Conditions: No randomization (ideal scenario)
% - Expected: High success rate, low error

\subsection{Test B: Calibration Robustness}
% - Inject systematic camera offset: 2cm, 3cm, 5cm
% - Measure: Can policy compensate to reach true target?
% - Baseline comparison: MoveIt alone (no compensation)

\subsection{Test C: Sensor Noise Robustness}
% - Vary AprilTag detection noise: σ = 0.5mm, 2mm, 5mm
% - Measure: Tracking accuracy degradation vs noise level

\subsection{Test D: Dynamic Robustness}
% - Vary friction coefficients: ±10%, ±20%
% - Vary link masses: ±10%
% - Measure: Success rate vs parameter deviation
% Table: Test conditions summary
