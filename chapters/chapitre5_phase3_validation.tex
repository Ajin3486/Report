\section{Validation in Isaac Lab }
To assess the performance of our residual reinforcement learning approach, we first evaluated the learned policy directly in the Isaac Lab 
training environment. This controlled simulation setting allows us to precisely measure the improvement brought by the AI agent compared to
 a Open-Loop controller under localized perturbations.

\subsection{Experimental Control Modes: Open-Loop vs. AI Closed-Loop}
Before presenting the quantitative results, we distinguish the two control strategies evaluated in this validation phase:

\begin{itemize}
    \item \textbf{Open-Loop (Baseline):} The robot executes the trajectory theoretically planned by MoveIt. In a perfect simulation, this would be accurate. However, in our "Sim-to-Real" validation scenarios, we introduce calibration errors (offset between the camera and the robot base) and sensor noise. In open-loop, the robot blindly follows the plan without correcting for these discrepancies, leading to significant end-effector errors.
    
    \item \textbf{Closed-Loop RL (AI):} This is the proposed residual learning approach. The AI agent acts as a high-frequency feedback controller (60Hz). It continuously monitors the \textit{observed} state (proprioception + noisy vision) and the tracking error. Unlike the baseline, the AI "closes the loop" by injecting real-time joint corrections ($\Delta q$) to minimize the distance to the perceived target. This allows it to dynamically compensate for the static calibration errors and filter out high-frequency sensor noise.
\end{itemize}

The achieved success rate exceeds 95\% on a test set comprising random trajectories with perception noise and simulated calibration error. This validates that the agent has successfully learned to correct the trajectories from the MoveIt planner.

\begin{figure}[htbp]
    \centering
    \includegraphics[width=0.9\textwidth]{images/table_summary.png}
    \caption{Performance summary comparing open-loop and closed-loop RL approaches. The RL policy achieves 69.7\% improvement in mean error, 55.6\% in max error, and 71.4\% in standard deviation.}
    \label{fig:performance_summary}
\end{figure}

Figure~\ref{fig:performance_summary} presents quantitative metrics comparing open-loop and closed-loop RL approaches:
\begin{itemize}
    \item \textbf{Mean error}: 28.37mm $\rightarrow$ 8.58mm (69.7\% improvement)
    \item \textbf{Max error}: 52.03mm $\rightarrow$ 23.08mm (55.6\% improvement)
    \item \textbf{Standard deviation}: 15.36mm $\rightarrow$ 4.39mm (71.4\% improvement)
    \item \textbf{Speed}: Average 187.9 mm/s, peak 587.5 mm/s
\end{itemize}
All error metrics show $>$50\% improvement, confirming the effectiveness of the RL correction approach.

\begin{figure}[htbp]
    \centering
    \includegraphics[width=0.95\textwidth]{images/fig1_error_comparison.png}
    \caption{Tracking accuracy comparison: Open-Loop vs Closed-Loop RL. The red curve shows the open-loop position error (mean=28.4mm, max=52.0mm), while the green curve demonstrates the closed-loop RL correction (mean=8.6mm, max=23.1mm), representing a 70\% improvement.}
    \label{fig:error_comparison}
\end{figure}

Figure~\ref{fig:error_comparison} shows the position error norm over time. The red curve (open-loop) exhibits mean error of 28.4mm with peaks exceeding 52mm, particularly during rapid direction changes at t=1.0s and t=3.8s. The green curve (closed-loop RL) maintains mean error of 8.6mm with maximum error below 23mm, representing a 70\% improvement. The RL agent anticipates errors and attenuates disturbances without introducing oscillations.

\begin{figure}[htbp]
    \centering
    \includegraphics[width=0.95\textwidth]{images/fig2_ee_position_x.png}
    \caption{Trajectory tracking on X-axis. The blue dashed line represents the desired trajectory, the orange dotted line shows the observed trajectory with noise, and the green solid line indicates the actual trajectory executed by the RL agent.}
    \label{fig:trajectory_x}
\end{figure}

Figure~\ref{fig:trajectory_x} shows end-effector position along the X-axis. The blue dashed line represents the ideal reference trajectory, the orange dotted line shows noisy observations with sensor noise and calibration errors, and the green solid line indicates the actual executed trajectory. The RL agent tracks the reference with 5.3mm average error on X-axis, demonstrating robustness to observation noise by filtering rather than blindly following noisy measurements.

\begin{figure}[htbp]
    \centering
    \includegraphics[width=0.95\textwidth]{images/fig4_joint_j4_strategy.png}
    \caption{Joint J4 (wrist\_1) position comparison. Top: open-loop tracking shows deviation from reference. Bottom: closed-loop RL applies a bias of +0.63° on J4 to improve end-effector accuracy.}
    \label{fig:joint_j4}
\end{figure}

Figure~\ref{fig:joint_j4} reveals the RL correction strategy at joint level for J4 (wrist\_1). Top panel: open-loop tracking follows the reference perfectly but leads to end-effector errors. Bottom panel: the RL policy systematically applies a +0.63\textdegree\ bias to J4. The gap between the reference (blue dashed) and RL-corrected target (orange dotted) shows this proactive compensation. This consistent bias compensates for systematic errors (gravity, friction, calibration) and exploits kinematic redundancy to improve end-effector accuracy without compromising motion quality.

\begin{figure}[htbp]
    \centering
    \includegraphics[width=0.9\textwidth]{images/fig3_ee_speed.png}
    \caption{End-effector speed during trajectory execution. Average speed: 187.9 mm/s, maximum speed: 587.5 mm/s.}
    \label{fig:ee_speed}
\end{figure}

Figure~\ref{fig:ee_speed} shows the end-effector velocity profile with average speed of 187.9 mm/s and peak of 587.5 mm/s at t=2.0s. The smooth profile without high-frequency oscillations indicates stable control. The velocity modulation demonstrates an emergent speed-accuracy trade-off learned during training, where the agent adaptively adjusts speed based on trajectory requirements while maintaining safety limits.

\section{Limitations of this validation}
Although the results in Isaac Lab are promising, this validation remains a "Sim-to-Sim" evaluation within the same physics engine (PhysX). 
The main risk is that the policy may have learned to exploit simulation-specific artifacts (numerical errors, simplified friction model) that do not exist in reality.
To ensure true robustness before real-world deployment, it is crucial to test the policy in a physically different environment.





