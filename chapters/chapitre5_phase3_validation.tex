\section{Validation Isaac Lab \texorpdfstring{$\rightarrow$}{->} Isaac Sim}
Nous avons d'abord évalué la politique apprise directement dans l'environnement d'entraînement Isaac Lab, mais sur des scénarios jamais vus durant l'entraînement. Les résultats montrent une convergence rapide et stable.

\begin{figure}[htbp]
    \centering
    % \includegraphics[width=0.8\textwidth]{images/learning_curve.png} % À remplacer par votre courbe
    \caption{Courbes d'apprentissage : récompense moyenne par épisode (Placeholder).}
    \label{fig:learning_curve}
\end{figure}

Le taux de succès atteint dépasse les 95\% sur un ensemble de test comprenant des trajectoires aléatoires avec bruit de perception et erreur de calibration simulée. Cela valide que l'agent a bien appris à corriger les trajectoires du planificateur MoveIt.

\section{Limites de cette validation}
Bien que les résultats dans Isaac Lab soient prometteurs, cette validation reste une évaluation "Sim-to-Sim" au sein du même moteur physique (PhysX). 
Le principal risque est que la politique ait appris à exploiter des artefacts spécifiques à la simulation (erreurs numériques, modèle de friction simplifié) qui n'existent pas dans la réalité.
Pour garantir une véritable robustesse avant le déploiement réel, il est crucial de tester la politique dans un environnement physiquement différent.

\section{Ouverture : Validation Sim-to-Sim (Isaac \texorpdfstring{$\rightarrow$}{->} MuJoCo)}
Pour renforcer notre validation, nous avons exporté la politique entraînée vers un second simulateur : MuJoCo.

\subsection{Pourquoi MuJoCo ?}
MuJoCo utilise un solveur de dynamique différent de celui d'Isaac Lab (coordonnées généralisées vs coordonnées maximales/réduites de PhysX) et un modèle de contact distinct. 
Si la politique fonctionne dans MuJoCo sans réentraînement, cela démontre une grande capacité de généralisation et suggère une bonne transférabilité vers le robot réel.

\subsection{Méthodologie de transfert}
Le transfert a nécessité la conversion du modèle du robot :
\begin{itemize}
    \item Conversion du format URDF (utilisé par ROS 2 et Isaac Lab) vers le format MJCF (spécifique à MuJoCo).
    \item Vérification de la correspondance des repères et des limites articulaires.
\end{itemize}

\subsection{Tests de robustesse}
Dans MuJoCo, nous avons soumis l'agent à des perturbations supplémentaires :
\begin{itemize}
    \item Ajout de masses inconnues sur l'effecteur (simulant une charge utile).
    \item Introduction d'erreurs de calibration plus sévères que lors de l'entraînement.
\end{itemize}
