\section{Physical Setup and ROS2 Pipeline}

\begin{figure}[h]
\centering
\includegraphics[width=0.5\textwidth]{images/Robot_reel.jpg}
\caption{Real-world setup: UR10 manipulator, camera, AprilTag markers, and workspace}
\label{fig:real_setup}
\end{figure}

The physical system consists of a Universal Robots UR10 manipulator (6 degrees of freedom), a webcam for perception, and AprilTag fiducial markers placed in a defined workspace (X: [0.7m, 1.0m], Y: [-0.2m, 0.2m], Z: 0.2m fixed height). This setup provides a controlled environment for training data generation and real-world validation.

\textbf{Perception and Localization.} The camera detects AprilTag markers and estimates their 3D positions relative to the camera. To enable motion planning, we establish a fixed world coordinate system anchored to one reference tag (Tag 1) with a vertical offset of 21.5cm to align with the robot's base frame. All other tags are then localized in this consistent world frame. This approach avoids drift: as long as the reference tag is visible, the coordinate system remains stable.

\textbf{Motion Planning Pipeline.} The system uses MoveIt2 to generate collision-free trajectories. Given a target position in world coordinates, MoveIt2 first computes inverse kinematics to find a valid starting joint configuration, then plans a smooth Cartesian path that moves the end-effector in a straight line while respecting joint limits and avoiding collisions.

\textbf{Training Data Generation.} To create supervised learning data, we generate random straight-line trajectories between points in the workspace. Each trajectory is planned by MoveIt2, then resampled to a fixed length of 64 timesteps (5.33 seconds at 12Hz control frequency) . For real-world validation, we also generate trajectories between actual physical tag positions, ensuring the dataset reflects real geometric constraints.

\textbf{Model Consistency.} Both ROS2 and Isaac Lab use identical robot models, ensuring centimeter-precise kinematic alignment. This consistency is critical: any mismatch in link lengths or joint axes would cause policies trained in simulation to fail on the real robot.

\section{Isaac Lab Digital Twin}
% To complete (~1.5 pages):
