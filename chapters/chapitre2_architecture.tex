% Chapter 2: System Architecture (~4 pages)

\section{Physical Setup and ROS2 Pipeline}
% To complete (~1 page):
% - Hardware: UR10 robot + Camera + AprilTags in defined workspace
% - ROS2 architecture: Camera node → TF (AprilTag to base) → MoveIt
% - Trajectory generation: Random straight lines in workspace
% - URDF: Same model in ROS2 and Isaac Lab (centimeter-precise alignment)
% Photo: Real setup (or CAD model)
% Diagram: ROS2 node architecture

\section{Isaac Lab Digital Twin}
% To complete (~1.5 pages):
% - Why Isaac Lab: GPU parallelization (4096 envs), faster than Isaac Sim
% - Synthetic vision: Math-based perception instead of image rendering
%   * Position_observed = Position_true + Gaussian_noise
%   * Equivalent to AprilTag detection but computationally efficient
% - Domain randomization during training:
%   * Sensor noise: σ_detection = [0.5mm, 2mm]
%   * Calibration error: Camera offset = [0, 3cm]
%   * Dynamics: Friction/damping ±15%, mass ±5%
%   * Latency: Action delay = [0, 50ms]
% Table: Randomization parameters with ranges
% Diagram: Isaac Lab architecture (4096 parallel environments)

\section{PPO Implementation}
% To complete (~1.5 pages):
% - Network: MLP (64-64 neurons, ReLU activation)
% - Observation (dim=18): Joint pos/vel (12), error vector to target (3), target position (3)
% - Action (dim=6): Position delta (corrections to MoveIt trajectory)
% - Reward function:
%   R = -α||error_pos|| - β||acceleration|| - γ·out_of_workspace
%   (α=10, β=0.1, γ=100)
% - Hyperparameters: lr=3e-4, batch=2048, epochs=10, horizon=512
% Table: Hyperparameters
% Equation: Reward function
